\documentclass[8pt, light]{oist_presentation} 

\usepackage{graphicx}
\usepackage{booktabs}
\usepackage{setspace}
\doublespacing

\mode<presentation> {
\usetheme{default}
\definecolor{OISTcolor}{rgb}{0.65,0.16,0.16}
\setbeamertemplate{itemize item}{\color{\templatecolor}$\bullet$}
\usepackage{helvet}
\renewcommand{\familydefault}{\sfdefault}
\setbeamertemplate{navigation symbols}{}
\setbeamertemplate{footline}
{\begin{minipage}{150mm} \vspace{-3.5 mm} \hfill \insertframenumber \end{minipage}}
}

\title[Short title]{\color{\templatecolor} 
Report on 8-bit MIPS Design and Simulation
}

\author{Contributor Student ID \linebreak \linebreak 1705068 \linebreak 1705069 \linebreak 1705070 \linebreak 1705071 \linebreak 1705074 \linebreak \linebreak  \textbf{Section: B(1)} \linebreak \textbf{Group: 3}} % Your name
\date{\today} % Date, leave empty, use subtitle instead

\begin{document}

\setbeamertemplate{background}{\includetitlebackground}

\begin{frame}[plain]
\vspace*{1.55cm}
\end{frame}

\setbeamertemplate{background}{\includebackground}

% ---------------------- Begin Introduction ---------------------- % 
\begin{frame}
\frametitle{Overview}
\tableofcontents
\end{frame}

%----------------------------------------------------------------------------------------
%	PRESENTATION SLIDES
%----------------------------------------------------------------------------------------

\section{Introduction}

\begin{frame}
\frametitle{Introduction}
    MIPS is a reduced computer instruction set architecture which defines the
interface between a user and a microprocessor. Each instruction execute in one or more clock cycle. As, the instruction a user write is not understandable by computer and each instruction can be different than others' so designing a microprocessor is can be cumbersome. MIPS reduced the hassle using three different principle ”Simplicity favours regularity” and "".
\linebreak

    In this assignment we designed a 8-bit MIPS which implements a custom
instruction set.

\end{frame}

%------------------------------------------------
\section{Instruction Set}
\begin{frame}
\frametitle{Instruction Set}
\begin{table}
\begin{tabular}{| c || c || c |}
\hline
\textbf{Instruction Type} & \textbf{Instruction} & \textbf{Category}\\
\hline
I & addi & Arithmetic \\
\hline
I & subi & Arithmetic \\
\hline
I & andi & Logic \\
\hline
I & ori & Logic \\
\hline
I & sw & Memory \\
\hline
I & lw & Memory \\
\hline
I & beq & Control-conditional \\
\hline
I & bneq & Control-conditional \\
\hline
R & add & Arithmetic \\
\hline
R & sub & Arithmetic \\
\hline
R & and & Logic \\
\hline
R & or & Logic \\
\hline
R & sll & Logic \\
\hline
R & srl & Logic \\
\hline
R & nor & Logic \\
\hline
J & j & Control-unconditional \\
\hline
\end{tabular}
\caption{IC used with count}
\end{table}
\end{frame}

%------------------------------------------------
\section{Instruction Set}
\begin{frame}
\frametitle{Instruction Format}
\begin{table}
\begin{tabular}{c  | c || c || c |}


\end{tabular}
\end{table}

\end{frame}
%------------------------------------------------
\section{Block Diagram}
\subsection{Complete Block diagram of an 8-bit MIPS processor}
\begin{frame}
\frametitle{Blocks Diagram}

\end{frame}

%------------------------------------------------

\subsection{Block diagrams of the main components}
\subsubsection{Instruction memory with PC}
\subsubsection{Register file}
\subsubsection{Data memory with the Stack}
\subsubsection{Control unit}
\begin{frame}
\frametitle{Block diagrams of the main components}
\begin{columns}[c]

\column{.45\textwidth}
\textbf{Heading}
\begin{enumerate}
\item Statement
\item Explanation
\item Example
\end{enumerate}

\column{.5\textwidth} % Right column and width
Lorem ipsum dolor sit amet, consectetur adipiscing elit. Integer lectus nisl, ultricies in feugiat rutrum, porttitor sit amet augue. Aliquam ut tortor mauris. Sed volutpat ante purus, quis accumsan dolor.

\end{columns}
\end{frame}

%------------------------------------------------
\section{Push and Pop Instruction}
%------------------------------------------------

\begin{frame}
\frametitle{Push and Pop Instruction}
\begin{table}
\end{table}
\end{frame}

%------------------------------------------------

%------------------------------------------------

\section{IC used with Count}
\begin{frame}
\frametitle{IC count}
\end{frame}

\section{Simulator}
\begin{frame}
\frametitle{Simulator}
\textbf{Logisim 2.7.1}
\end{frame}

%------------------------------------------------

\section{Discussion}
\begin{frame}
\frametitle{Discussion}
\begin{theorem}[Mass--energy equivalence]
$E = mc^2$
\end{theorem}
\end{frame}



%----------------------------------------------------------------------------------------

\end{document} 